%Autor: Martin Hruska
%E-mail: xhrusk16@stud.fit.vutbr.cz
%Datum: 15.11.2013

\documentclass[a4paper]{article}
\usepackage[left=1.5cm, text={18cm, 25cm}, top=2.5cm]{geometry}
\usepackage[utf8]{inputenc}
\usepackage[czech]{babel}
\usepackage{amsthm}
\usepackage{amsfonts}
\usepackage{amsmath}
\usepackage{amssymb}
\usepackage[noend,noline,commentsnumbered,linesnumbered,czech]{algorithm2e}
\usepackage{tikz}
\usetikzlibrary{automata,arrows,topaths}
\renewcommand{\KwIn}[1]{\textbf{Vstup:} #1\\}
\renewcommand{\KwOut}[1]{\textbf{Výstup:} #1\\}

\usepackage{mdwlist}
\newcommand{\myuv}[1]{\quotedblbase #1\textquotedblleft}

\usepackage{atbegshi,picture}
\AtBeginShipout{\AtBeginShipoutUpperLeft{%
  \put(\dimexpr\paperwidth-1cm\relax,-1.5cm){\makebox[0pt][r]{\framebox{Martin Hruška, xhrusk16}}}%
}}


\title {1. úloha}
\begin {document}
\section*{\centering{1. příklad}}
Mějme Turingův stroj M rozhodující L fungující následujícím způsobem:
\begin{enumerate}
    \item Vstup bude na pásce M zapsán v~následujícím tvaru $\Delta a^j b^k c^l\Delta$, kde $j,k,l \in \mathbb{N}$.
    Pro zbytek příkladu uvažujme, že $n=j+k+l$
    \item Stroj přejde na začátek pásky (maximálně $n$ kroků).
    \item Stroj přejde doprava na první symbol z~množniny $\{a,b,c\}$.
    Pokud tímtom symbolem není $a$, pak provede abnormální ukončení a řetězec odmítne.
    Pokud není na pásce žádný symbol z~dané množiny, stroj příjme řetězec.
    V~tomto bodě se provede vždy maximálně $\frac{n}{3}$ kroků stroje.
    \item Stroj se nachází symbolu $a$. Tento symbol nahradí symbolem $\Delta$ (1 krok stroje)
    \item Stroj přejde na první symbol $b$.
    Pokud takový symbol na pásce není, provede abnormální ukončení.
    Stroj v~tomto bodu provede maximálně $\frac{n}{3}$ kroků.
    \item Stroj nahradí symbol $b$ za symbol $\Delta$ (1 krok stroje).
    \item Stroj přejde na první výskyt symbol $c$ (maximálně $\frac{n}{3}$ kroků).
    Pokud takovýto symbol neexistuje, provede abnormální ukončení výpočtu.
    \item Vrať se na bod 2.
\end{enumerate}
Algoritmus pracuje v~cyklech (vznikají v~bodu 8),
jejichž maximální počet bude $\frac{n}{3}$, jelikož v~každém cyklu přepíše jeden symbol $a$ a skončit může, až přepíše
všechny tyto symboly (tedy po maximálně $\frac{n}{3}$ opakování) nebo v~případě odmítnutí dříve.
Stroj v~jednom cyklu provede maximálně $n+\frac{n}{3}+1+\frac{n}{3}+1+\frac{n}{3}=2*n+2$ kroků, celý výpočet pak zabere
maximálně $\frac{n}{2}*(2*n+2)=n^2+n=O(n^2)$ kroků. Horní odhad složitosti výpočtu tedy je $O(n^2)$.

Zmíněný algoritmus pracuje pouze na části pásky dané vstupem, jelikož přepisuje jeho jednotlivé symboly.
Horní odhad prostorové složitosti je tedy $|a^j b^k c^l|=n=O(n)$. 

\section*{\centering{2. příklad}}
RAM program řešící zadanou úlohu je následující:
\begin{enumerate}
    \item LOAD =0
    \item STORE 3
    \item READ 2
    \item STORE 2
    \item READ 1
    \item SUB 2
    \item JNEG 14
    \item STORE 1
    \item LOAD 3
    \item ADD =1
    \item STORE 3
    \item LOAD 1
    \item JUMP 6
    \item HALT
\end{enumerate}

Při analýze uniformní složitosti přepokládáme, že každá instrukce má stejnou časovou složitost.
V~bodě 13 vzniká cyklus, který má časovou složitost $8$. Program mimo cyklus má časovou složitost $6$.
Cyklus se neprovede více než $n1$-krát, a tedy dostáváme horní odhad časové složitosti $8*2^n+6=O(2^{n})$,
protože $n1 \leq 2^n$, kde $n$ je délka vstupu.
Horní odhad prostorové složitosti je pak $2+4=6$, kde $2$ je počet vstupních hodnot a $4$ je počet použitých registrů. 

Při analýzy logaritmické složitosti víme, že žádná instrukce nebude v~programu pracovat s~číslem větším než $2^n$
(kde $n$ je opět délka vstupu).
Časová složitost algoritmu bude tudíž $8*log(2^n)+6*log(2^n)$ a vzhledem k~počtu opakování cyklu, které zůstává
stejné jako v~předchozím případu, bude horní odhad složitosti algoritmu v~$O(log(2^n)*2^n)$.
Jelikož žádná z~instrukcí nepracuje s~číslem delším než $n$, tak prostorová složitost bude v~$O(n)$.

\section*{\centering{3. příklad}}
\begin{enumerate}
    \item Nechť je $L$ regulární jazyk a $L \in NTIME(f(n))$
    \item Jazyk $L$ je regulární, tudíž ho lze popsat pomocí deterministického konečného automatu\\ $M_{KA}=(Q_{KA},\Sigma, \delta_{KA}, q_{0_{KA}}, F_{KA})$.
    \item Tento automat pak lze simulovat pomocí deterministického Turingova stroje Turingova stroje $M=(Q, \Sigma, \Gamma, \delta, q_0, q_f)$, kde:
    \begin{itemize}
        \item $Q=Q_{KA}\cup {q_f}$
        \item $\Gamma = \Sigma \cup {\delta}$
        \item $\delta$ je zkonstruovaná následovně:
        \begin{itemize}
            \item $\forall q \in Q \forall a \in \Sigma: (q,R) = \delta(p,a) \Leftrightarrow q=\delta_{KA}(p,a)$
            \item $\forall q \in F_{KA}: (q_f,\Delta) = \delta(q,\Delta)$
        \end{itemize}
        \item $q_0 = q_{0_{KA}}$
        \item $q_f \in Q \wedge q_f \notin Q_{KA}$
    \end{itemize}
    \item Počáteční konfigurace $M$ je $(q_0,\underline{a_1} a_2 a_3\ldots a_n\Delta^{\omega})$, kde $a_1a_2a_3\ldots a_n \in \Sigma^*$
    \item Časová složitost stroje $M$ je funkce $T_M(n)=n+1$, a tedy stroj libovolný vstup zpracuje nanejvýš po $n+1$ krocích.
    Z~jeho definice je zřejmé (by rapid handwaving), že musí projít přes první symbol řetězce $w$ až po jeho poslední symbol a poté teprve
    přejde do koncového stavu.
    V~každém kroku, kdy má pod hlavou symbol z~$\Sigma$ udělá pohyb hlavou doprava nebo se zasekne a tím řetězec odmítne.
    Takových to kroků udělá nanejvýše $n$, kde $|\omega| = n$.
    Pokud je pod hlavou symbol $\Delta$ může buď přejít do koncového stavu nebo se zaseknout, a tak řetězec odmítnout.
    Tavovýto krok může být učiněn maximálně jeden.
    Dohromady ted může provést nejvýše $n+1$ kroků.
    \item Bylo ukázáno, že pro libovolný jazyk lze sestavit deterministický Turingův s~$T_M(n)=n+1$, a proto takový jazyk
    bude patřit $DTIME(O(n))$.
    \item Jelikož $DTIME(f(n)) = \{L\,|\, \emph{existuje TM }M' \emph{ rozhodující }L\emph{ s~časovou složitostí } T_{M'}(n) \leq f(n)\}$ a zároveň
    víme, že $\forall n \in N.f(n) \geq n$ a $T_M(n)=n$, tedy dohromady $T_M(n)=n \leq f(n)$, a tudíž $L\in DTIME(f(n))$ $\square$
\end{enumerate}
\end{document}
