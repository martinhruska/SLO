%Autor: Martin Hruska
%E-mail: xhrusk16@stud.fit.vutbr.cz
%Datum: 15.11.2013

\documentclass[a4paper]{article}
\usepackage[left=1.5cm, text={18cm, 25cm}, top=2.5cm]{geometry}
\usepackage[utf8]{inputenc}
\usepackage[czech]{babel}
\usepackage{amsthm}
\usepackage{amsfonts}
\usepackage{amsmath}
\usepackage{amssymb}
\usepackage{verbatim}
\usepackage[noend,noline,commentsnumbered,linesnumbered,czech]{algorithm2e}
\usepackage{tikz}
\usetikzlibrary{automata,arrows,topaths}
\renewcommand{\KwIn}[1]{\textbf{Vstup:} #1\\}
\renewcommand{\KwOut}[1]{\textbf{Výstup:} #1\\}

\usepackage{mdwlist}
\newcommand{\myuv}[1]{\quotedblbase #1\textquotedblleft}

\usepackage{atbegshi,picture}
\AtBeginShipout{\AtBeginShipoutUpperLeft{%
  \put(\dimexpr\paperwidth-1cm\relax,-1.5cm){\makebox[0pt][r]{\framebox{Martin Hruška, xhrusk16}}}%
}}


\title {1. úloha}
\begin {document}
\section*{\centering{1. příklad}}
Mějme Turingův stroj M rozhodující L fungující následujícím způsobem:
\begin{enumerate}
    \item Vstup bude na pásku stroje M zapsán v~následujícím tvaru $\Delta a^j b^k c^l\Delta$, kde $j,k,l \in \mathbb{N}$.
    Pro zbytek příkladu uvažujme, že $n=j+k+l$
    \item Stroj přejde na začátek pásky (maximálně $n$ kroků, tedy složitost $O(n)$).
    \item Stroj přejde doprava na první symbol z~množniny $\{a,b,c\}$.
    Pokud tímtom symbolem není $a$, pak provede abnormální ukončení a řetězec odmítne.
    Pokud není na pásce žádný symbol z~dané množiny, stroj příjme řetězec.
    Tento krok má složitost $O(n)$.
    \item Stroj se nachází na symbolu $a$. Tento symbol nahradí symbolem $\Delta$ (1 krok stroje)
    \item Stroj přejde na první symbol $b$.
    Pokud takový symbol na pásce není, provede abnormální ukončení.
    Složitost kroku je $O(n)$.
    \item Stroj nahradí symbol $b$ za symbol $\Delta$ (1 krok stroje).
    \item Stroj přejde na první výskyt symbol $c$ (složitost $O(n)$).
    Pokud takovýto symbol neexistuje, provede abnormální ukončení výpočtu.
    \item Vrať se na bod 2.
\end{enumerate}
Algoritmus pracuje v~cyklech (vznikají v~bodu 8),
jejichž maximální počet bude $\frac{n}{3}$, jelikož v~každém cyklu přepíše jeden symbol $a$ a skončit může, až přepíše
všechny tyto symboly (tedy po maximálně $\frac{n}{3}$ opakování) nebo v~případě odmítnutí dříve.
Jeden cyklus v algoritmu má složiost $O(n)+O(n)+1+O(n)+1+O(n)=O(n)$, celý výpočet pak
$\frac{n}{2}*O(n)=O(n^2)$. Horní odhad složitosti výpočtu tedy je $O(n^2)$.

Popsaný algoritmus pracuje pouze na části pásky dané vstupem, jelikož přepisuje jeho jednotlivé symboly.
Horní odhad prostorové složitosti je tedy $|a^j b^k c^l|=n=O(n)$. 

\section*{\centering{2. příklad}}
RAM program řešící zadanou úlohu je následující:
\begin{enumerate}
    \item LOAD =0
    \item STORE 3
    \item READ 2
    \item STORE 2
    \item READ 1
    \item SUB 2
    \item JNEG =14
    \item STORE 1
    \item LOAD 3
    \item ADD =1
    \item STORE 3
    \item LOAD 1
    \item JUMP 6
    \item LOAD 3
    \item HALT
\end{enumerate}

Při analýze uniformní složitosti přepokládáme, že každá instrukce má stejnou časovou složitost.
V~bodě 13 vzniká cyklus, který obsahuje $8$ instrukcí a tedy má časovou složitost $8$. Program mimo cyklus má časovou složitost $7$.
Cyklus se neprovede více než $n1$-krát, a tedy dostáváme horní odhad časové složitosti $8*2^n+7=O(2^{n})$,
protože $n1 \leq 2^n$, kde $n$ je délka vstupu.
Horní odhad prostorové složitosti je pak $2+4=6$, kde $2$ je počet vstupních hodnot a $4$ je počet použitých registrů. 

Při analýze logaritmické složitosti víme, že žádná instrukce nebude v~programu pracovat s~číslem větším než $2^n$
(kde $n$ je opět délka vstupu).
Cena instrukce bude tedy nejvýše $log(2^n)$.
Počet cyklů vykonaných programem zůstává stejný a dohromady tedy dostáváme následující horní odhad časové složitosti
$2^n*(8*log(2^n))+7*log(2^n)=O(log(2^n)*2^n)$.
Žádná z~instrukcí nepracuje s~číslem delším než $n$ a žádné delší číslo nemůže být uloženo ani do registrů, jelikož výsledek výpočtu
nepřesáhne hodnotu $n_1$. Horní odhad prostorové složitosti tedy bude $n+4*O(n)=O(n)$.

\section*{\centering{3. příklad}}
\begin{enumerate}
    \item Nechť je $L$ regulární jazyk a $L \in NTIME(f(n))$
    \item Jazyk $L$ je regulární, tudíž ho lze popsat pomocí deterministického konečného automatu.%\\ $M_{KA}=(Q_{KA},\Sigma, \delta_{KA}, q_{0_{KA}}, F_{KA})$.
    Automat příjme (nebo odmítne) řetězec vždy nejvýše po přečtení celého vstupu (tedy po $n$ krocích),
    jelikož se vždy po přečtení symbolu přesune na pásce o~jedno místo doprava.
    \item Mějme Turingův stroj $M$ pracující stejně jako deterministický automat, tudíž bude také deterministický, vstupní řetězec zpracuje
    nejvýše po $n$ krocích a přijímá jazyk $L$
    \begin{comment}
    \item Tento automat pak lze simulovat pomocí deterministického Turingova stroje $M=(Q, \Sigma, \Gamma, \delta, q_0, q_f)$, kde:
    \begin{itemize}
        \item $Q=Q_{KA}\cup \{q_f\}$
        \item $\Gamma = \Sigma \cup \{\Delta\}$
        \item $\delta$ je zkonstruovaná následovně:
        \begin{itemize}
            \item $\forall q \in Q\, \forall a \in \Sigma: (q,R) = \delta(p,a) \Leftrightarrow q=\delta_{KA}(p,a)$
            \item $\forall q \in F_{KA}: (q_f,\Delta) = \delta(q,\Delta)$
        \end{itemize}
        \item $q_0 = q_{0_{KA}}$
        \item $q_f \in Q \wedge q_f \notin Q_{KA}$
    \end{itemize}
    \item Počáteční konfigurace $M$ je $(q_0,\underline{a_1} a_2 a_3\ldots a_n\Delta^{\omega})$, kde $a_1a_2a_3\ldots a_n \in \Sigma^*$
    \item Časová složitost stroje $M$ je funkce $T_M(n)=n+1$, a tedy stroj libovolný vstup zpracuje nanejvýš po $n+1$ krocích.
    Z~jeho definice je zřejmé, že musí projít přes první symbol řetězce $w$ až po jeho poslední symbol a poté teprve
    přejde do koncového stavu.
    V~každém kroku, kdy má pod hlavou symbol z~$\Sigma$ udělá pohyb hlavou doprava nebo se zasekne a tím řetězec odmítne.
    Takovýchto kroků udělá nanejvýše $n$, kde $|\omega| = n$.
    Pokud je pod hlavou symbol $\Delta$ může buď přejít do koncového stavu nebo se zaseknout, a tak řetězec odmítnout.
    Tento krok může být učiněn maximálně jeden.
    Dohromady tedy může provést nejvýše $n+1$ kroků.
    \end{comment}

    \item Turingův stroj má tedy časovou složitost $T_M(n)=n$, a proto jazyk $L$ bude patřit do $DTIME(O(n))$.
    \item Jelikož $DTIME(f(n)) = \{L\,|\, \emph{existuje TM }M' \emph{ rozhodující }L\emph{ s~časovou složitostí } T_{M'}(n) \leq f(n)\}$ a zároveň
    víme, že $\forall n \in N.f(n) \geq n$ a $T_M(n)=n$, tedy dohromady $T_M(n)=n \leq f(n)$, a tudíž $L\in DTIME(f(n))$ $\square$
\end{enumerate}
\end{document}
