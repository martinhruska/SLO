%Autor: Martin Hruska
%E-mail: xhrusk16@stud.fit.vutbr.cz
%Datum: 16.5.2014

\documentclass[a4paper]{article}
\usepackage[left=1.5cm, text={18cm, 25cm}, top=2.5cm]{geometry}
\usepackage[utf8]{inputenc}
\usepackage[czech]{babel}
\usepackage{amsthm}
\usepackage{amsfonts}
\usepackage{amsmath}
\usepackage{amssymb}
\usepackage{verbatim}
\usepackage{tikz}
\usetikzlibrary{automata,arrows,topaths}

\usepackage{mdwlist}
\newcommand{\myuv}[1]{\quotedblbase #1\textquotedblleft}
\newcommand{\fls}{false}
\newcommand{\tre}{true}
\newcommand{\prt}{OPT\_PARTITION}


\usepackage{atbegshi,picture}
\AtBeginShipout{\AtBeginShipoutUpperLeft{%
  \put(\dimexpr\paperwidth-1cm\relax,-1.5cm){\makebox[0pt][r]{\framebox{Martin Hruška, xhrusk16}}}%
}}


\title {4. úloha}
\begin {document}
\section*{\centering{1. příklad}}
Důkaz provedeme sporem s předpokladem, že \emph{klika} je $\epsilon$-aproximační:
\begin{itemize}
    \item Graf, jehož uzel \emph{max} má stupeň je $k$, bude mít kliku velkou maximálně $k+1$ a to v případě,
    že všechny hrany uzlu \emph{max} spojují uzel s dalšími uzly, které patří do kliky.
    $k+1$ je tedy pro tyto grafy optimální řešení problému klika.
    \item Uvažujme, že uzel \emph{max} má stupeň alespoň $k \geq 2$.
    Jeho klika bude mít tedy velikost maximálně $k+1$,
    ale algoritmus \emph{klika} najde řešení o velikosti $2$.
    To nastane v případě, kdy na začátku bylo vybráno mezi uzly $A,B$, jejichž stupeň je $k$ a $A>B$,
    ale $A$ patří do kliky pouze o velikosti $2$.
    Menší klika existovat nemůže, protože by $A$ neměl stupeň $k \geq 2$.
    \item Předpokládejme nyní, že je klika $\epsilon$-aproximační, kde $\epsilon <1$, pak by muselo platit:
    $$ \frac{k+1-2}{k+1} < \epsilon$$
    $$ \frac{k-1}{k+1} < \epsilon$$
    Což ale pro žádné $\epsilon < 1$ neplatí, protože hodnota $\frac{k-1}{k+1}$ se bude pro rostoucí $k$
    neustále přibližovat hodnotě $1$ a nelze ji tedy ohraničit žádným číslem menším než $1$.
\end{itemize}
\section*{\centering{2. příklad}}
\end{document}

