%Autor: Martin Hruska
%E-mail: xhrusk16@stud.fit.vutbr.cz
%Datum: 16.5.2014

\documentclass[a4paper]{article}
\usepackage[left=1.5cm, text={18cm, 25cm}, top=2.5cm]{geometry}
\usepackage[utf8]{inputenc}
\usepackage[czech]{babel}
\usepackage{amsthm}
\usepackage{amsfonts}
\usepackage{amsmath}
\usepackage{amssymb}
\usepackage{verbatim}
\usepackage{tikz}
\usetikzlibrary{automata,arrows,topaths}

\usepackage{mdwlist}
\newcommand{\myuv}[1]{\quotedblbase #1\textquotedblleft}
\newcommand{\fls}{false}
\newcommand{\tre}{true}
\newcommand{\sbs}{\# SUBSET\ SUM}


\usepackage{atbegshi,picture}
\AtBeginShipout{\AtBeginShipoutUpperLeft{%
  \put(\dimexpr\paperwidth-1cm\relax,-1.5cm){\makebox[0pt][r]{\framebox{Martin Hruška, xhrusk16}}}%
}}


\title {4. úloha}
\begin {document}
\section*{\centering{1. příklad}}
Důkaz provedeme sporem s~předpokladem, že \emph{klika} je $\epsilon$-aproximační:
\begin{itemize}
    \item Graf, jehož uzel \emph{max} má stupeň je $k$, bude mít kliku velkou maximálně $k+1$ a to v~případě,
    že všechny hrany uzlu \emph{max} spojují uzel s~dalšími uzly, které patří do kliky.
    Klika o~velikosti $k+1$ je tedy pro tyto grafy optimálním řešením.
    \item Uvažujme, že uzel \emph{max} má stupeň alespoň $k \geq 2$ (jinak algoritmus najde vždy optimální řešení, protože
    graf nebude mít kliku větší než 2).
    Jeho klika bude mít tedy velikost maximálně $k+1$,
    ale algoritmus \emph{klika} najde řešení o~velikosti $2$.
    To nastane v~případě, kdy na začátku bylo vybráno mezi uzly $A,B$, jejichž stupeň je $k$ a $A>B$,
    ale $A$ patří do kliky pouze o~velikosti $2$.
    Menší klika existovat nemůže, protože by $A$ neměl stupeň $k \geq 2$.
    \item Předpokládejme nyní, že je klika $\epsilon$-aproximační, kde $\epsilon <1$, pak by muselo platit:
    $$ \frac{k+1-2}{k+1} < \epsilon$$
    $$ \frac{k-1}{k+1} < \epsilon$$
    Což ale pro žádné $\epsilon < 1$ neplatí, protože hodnota $\frac{k-1}{k+1}$ se bude pro rostoucí $k$
    neustále přibližovat hodnotě $1$ a nelze ji tedy ohraničit žádným číslem menším než $1$ $\square$
\end{itemize}

\section*{\centering{2. příklad}}
Napřed ukážeme, že $\sbs \in \#P$:
\begin{itemize}
    \item Relace $R$ spojená s~$\sbs\ $ je polynomiálně vyvážená, protože pokud
    $(X,Y) \in R$, kde $X=(X',w,W)$, $X'$ je množina položek, $w$ množina jejich vah a $W$ požadovaná váha řešení,
    tak $Y \subseteq X'$ a tedy $|Y| \leq |X'| \leq |X|$.
    \item Je-li dáno $X$ a $Y$, tak lze polynomiálně rozhodnout, zdali $(X,Y)\in R$ tak,
    že Turingův stroj sečte váhy všech prvků $Y$ a pokud jsou rovny $W$, tak $(X,Y)\in R$.
\end{itemize}
Nyní přistupme k~důkazu, že $\sbs$\ je NP-úplný.
Důkaz bude proveden redukcí ze $\#SAT$ (přesněji $\# 3-SAT$) následujícím způsobem:
\begin{itemize}
 \item Základem této redukce je redukce $SAT$ na $SUBSET\ SUM$ uvedená přednáškách.
 Tato redukce je část $R$ udávající zobrazení případů problému $\# SAT$ na případy problému $\sbs$.
 Následující notace je stejná jako na přednáškách.
 \item Nyní bude ukázáno jak pracuje funkce $S$, tedy jak lze pro každé řešení $\sbs$\ dostat zpět řešení $\# SAT$.
 \begin{itemize}
    \item Pokud má problém $SUBSET\ SUM$ řešení, tak vybraná podmnožina obsahuje buď $t_i$ nebo $f_i$, kde $i\in\{0,\ldots,n-1\}$ a
    $n$ je počet proměnných redukovaného $SAT$ problému, protože jinak by prvních $n$ cifer v~čísle udávajícím váhu
    podmnožiny nemělo hodnotu právě $1$, což je nezbytné vzhledem k~tomu, že $W$ má vždy hodnotu $1_0,\ldots,1_{n-1}3_0, \ldots,
    3_{k-1}$, kde $k$ je počet klausulí případu $SAT$ problému.
    Pokud je tedy v~podmnožině $t_i$, tak proměnná $x_i$ v~$SAT$ problému bude ohodnocena jako true, pokud $f_i$, tak jako false.
    Pro každé řešení $SUBSET\ SUM$ tedy dostáváme ohodnocení pro $SAT$ problém.
    Pokud tedy řešení $SUBSET\ SUM$ obsahuje jiný výběr prvků $t_i, f_i$ dostáváme jiné ohodnocení, a tedy pro každé
    ohodnocení $SAT$ problému dostaneme odlišné řešení $SUBSET\ SUM$, a tedy je identický počet ohodnocení $SAT$ problému
    jako řešení $SUBSET\ SUM$.
 \end{itemize}
Uvedená redukce je tedy šetrná, a tudíž je možné provést redukci ze $\#SAT$ na $\sbs$ $\square$
\end{itemize}
\end{document}
