%Autor: Martin Hruska
%E-mail: xhrusk16@stud.fit.vutbr.cz
%Datum: 27.4.2014

\documentclass[a4paper]{article}
\usepackage[left=1.5cm, text={18cm, 25cm}, top=2.5cm]{geometry}
\usepackage[utf8]{inputenc}
\usepackage[czech]{babel}
\usepackage{amsthm}
\usepackage{amsfonts}
\usepackage{amsmath}
\usepackage{amssymb}
\usepackage{verbatim}
\usepackage{tikz}
\usetikzlibrary{automata,arrows,topaths}

\usepackage{mdwlist}
\newcommand{\myuv}[1]{\quotedblbase #1\textquotedblleft}
\newcommand{\fls}{false}
\newcommand{\tre}{true}
\newcommand{\prt}{OPT\_PARTITION}


\usepackage{atbegshi,picture}
\AtBeginShipout{\AtBeginShipoutUpperLeft{%
  \put(\dimexpr\paperwidth-1cm\relax,-1.5cm){\makebox[0pt][r]{\framebox{Martin Hruška, xhrusk16}}}%
}}


\title {3. úloha}
\begin {document}
\section*{\centering{1. příklad}}
\textbf{a)} Předpokládejme, že $n$ je délka vstupu.
Časová složitost výpočtu následující:
\begin{itemize}
    \item Cyklus pro náhodné ohodnocení proměnných se opakuje maximálně $n$-krát, jelikož počet proměnných nebude nikdy větší než
     délka vstupní formule.
    Tedy složitosti cyklu je $O(n)$.
    \item Vyhodnocení formule lze provést v~lineárním čase $O(n)$.
    V~lineárním průchodu jsou vyhodnocovány jednotlivé klausule (a literály) a pokud je některá z~nich vyhodnocena jako \fls, tak je
    je celá formule vyhodnocena jako \fls.
    Pokud jsou všechny \tre, tak je celá formule ohodnocena jako \tre.
\end{itemize}
Celková složitost je tedy $O(n)+O(n)=O(n)$.
\\

\noindent\textbf{b)} Pro funkci \emph{SAT} (respektive pro TS ji realizující) platí následující:
\begin{itemize}
    \item Pokud funkce vrací \tre (stroj příjme), tak je formule určitě splnitelná, jelikož bylo nalezeno příslušné ohodnocení.
    \item Pokdu funkce vrací \fls (stroj zamítne), tak je formule nesplitelná s~pravděpodobností $\frac{2^m-x}{2^m}$, kde $m$ je počet proměnných a
    $x$ je počet ohodnocení, pro něž je vstupní formule splnitelná.
\end{itemize}
Je zřejmé, že pravděpodobnost špatné odpovědi (ta je rovna $1-\frac{1}{2^m}$) roste s~počtem proměnných ve formuli a
klesá s~počtem ohodnocení, při nichž je daná formule splnitelná.
Pravděpodobnost špatné odpovědi tedy není nezávislá na vstupní formuli.
\\

\noindent\textbf{c)} Dle definice z~b) stroj odmítne všechny nesplitelné SAT formule, jelikož při libovolném náhodně vygenerované ohodnocení nebude
nikdy formule platit.
Ovšem existují splnitelné formule s~jediním pravdivým ohodnocením, v~jejichž případě bude $\frac{2^m-1}{2^m}$ výpočtů neakceptujících,
což je pro $m>1$ méně než $\frac{1}{2}$, a tudíž SAT problém nepatří do třídy RP.

\pagebreak

\section*{\centering{2. příklad}}
Napřed ukažme, že $\prt$ patří do $NPO$:
\begin{itemize}
 \item $I=\{S,v\}$, kde $S$ je množina položek a $v$ jejich váhy (váhová funkce), patří zřejmě do $P$.
 \item Pro řešení $A$ platí, že $A\subseteq S$ a je tedy ohraničeno polynomem a ověření, zdali je $A$ podmnožinou $S$, lze spočítat v~polynomiálním čase,
       takže také lze ve stejném čase rozhodnout zdali $A$ je přípustné řešení.
 \item Cena řešení $A$ je ze zadání definována jako rozdíl dvou sum podmnožin množiny $S$,
 což lze spočítat deterministickým TS v~polynomiálním čase.
\end{itemize}
 Zároveň problém $\prt$ patří do $NPO$ a nikoliv do $PO$, jelikož asociovaný problém s~$\prt$ by pak musel být v~$P$, což neplatí.

 Nyní sporem dokážeme, že neexistuje absolutním optimalizační algoritmus pro problém $\prt$, pokud $P \neq NP$.
 Předpokládejme tedy, že takový algoritmus existuje.
 Označme tento algoritmus jako $O$ a předpokládejme, že pracuje s~chybou $k$.
 Na základě algoritmu $O$ navrhneme nový algoritmus, který pro libovolný případ $x$ problém $\prt$ najde jeho optimální řešení.
 Tento algoritmus bude pracovat tak, že libovolný případ $x$ přeformuluje na nový případ $x'$ problému $\prt$.
 Pro $x'$ platí, že $v'=(k+1)*v$, jinak je shodný s~$x$.
 Pro množinu přípustných řešení případu $x'$ platí, že $F(x')=F(x)$, jelikož opět vybíráme možné podmnožiny množiny položek $S$.
 Cena řešení u~$x'$ bude ovšem $(k+1)$-krát větší než u~$x$.
 Nyní pusťme na případu $x'$ algoritmus $O$ a dostaneme výsledek $O(x')$ s~chybou $k$ (tedy chybou algoritmu $O$).
 Platí tedy následující:
 $$ ||\sum_{x\in (x'\setminus O(x'))} v'(x)\ - \sum_{x\in O(x')} v'(x)| - OPT(x')| \leq k~$$
 Po dosazení s~ohledem na změnu váhové funkce a úpravách postupně dostaneme:
 $$ ||\sum_{x\in (x'\setminus O(x'))} (k+1)*v(x)\ - \sum_{x\in O(x')} (k+1)*v(x)| - (k+1)*OPT(x)| \leq k$$
 $$ |(k+1)*|\sum_{x\in (x'\setminus O(x'))} v(x)\ - \sum_{x\in O(x')} v(x)| - (k+1)*OPT(x)| \leq k$$
 $$ ||\sum_{x\in (x'\setminus O(x'))} v(x)\ - \sum_{x\in O(x')} v(x)| - OPT(x)| \leq \frac{k}{k+1}$$
 $$ ||\sum_{x\in (x'\setminus O(x'))} v(x)\ - \sum_{x\in O(x')} v(x)| - OPT(x)| \leq 0$$
 Poslendí krok je možný, jelikož pracujeme pouze s~přirozenými čísly.
 Dostáváme tedy, že $O(x')$ je optimálním řešením $x$, což je ovšem možné pouze když $P=NP$, což je spor s~předpokladem,
 a tedy nemůže existovat absolutní aproximační algoritmus $\square$
\end{document}

