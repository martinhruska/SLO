%Autor: Martin Hruska
%E-mail: xhrusk16@stud.fit.vutbr.cz
%Datum: 20.3.2013

\documentclass[a4paper]{article}
\usepackage[left=1.5cm, text={18cm, 25cm}, top=2.5cm]{geometry}
\usepackage[utf8]{inputenc}
\usepackage[czech]{babel}
\usepackage{amsthm}
\usepackage{amsfonts}
\usepackage{amsmath}
\usepackage{amssymb}
\usepackage{verbatim}
\usepackage{tikz}
\usetikzlibrary{automata,arrows,topaths}

\usepackage{mdwlist}
\newcommand{\myuv}[1]{\quotedblbase #1\textquotedblleft}
\newcommand{\rp}{RATE\_PARTITION}
\newcommand{\sbs}{SUBSET\ SUM}
\newcommand{\ltl}{L_t}

\usepackage{atbegshi,picture}
\AtBeginShipout{\AtBeginShipoutUpperLeft{%
  \put(\dimexpr\paperwidth-1cm\relax,-1.5cm){\makebox[0pt][r]{\framebox{Martin Hruška, xhrusk16}}}%
}}


\title {2. úloha}
\begin {document}
\section*{\centering{1. příklad}}
Napřed bude provedena analýza časové složitosti:
\begin{itemize}
    \item Inicializace jednotlivých datových struktur na řádcích $2,3$ bude provedena v~$n$ cyklech, kde $n$ je počet vrcholů vstupního grafu.
    V~každém cyklu budou provedeny $3$ kroky.
    Celkem tedy inicializace zabere $n*3$ kroků, což je v~$O(n)$.
    \item Nastavení hodnot pro první uzel v~prohledávání, které je provedeno na řádku $4$, a následná inicializace fronty (řádek $5$) mají
    složitost je $3+2=5$.
    \item Cyklus na řádku $6$ bude proveden maximálně $n$-krát, jelikož každý uzel se do fronty dostane právě jednou.
    Jedna iterace cyklu bude mít následující složitost:
    \begin{itemize}
        \item Výběr z~fronty na řádku $7$ bude trvat jeden krok výpočtu.
        \item Cyklus na řádku $8$ se bude opakovat maximálně $n-1$, jelikož z~každého uzlu grafu může vést hrana do každého z~ostatních uzlů.
        \item Tento cyklus obsahuje nejvýše $5$ kroků.
        Dohromady má tedy cyklus z~řádku $8$ horní odhad složitosti $5*(n-1)=O(n)$
        \item Operace na řádku $12$ má konstantní složitost $1$.
    \end{itemize}
    \item Jedna iterace cyklu na řádku $6$ bude mít tedy složitost $1+O(n)+1=O(n)$.
    Celý cyklus pak má horní odhad složitosti $n*O(n)=O(n^2)$.
\end{itemize}
Celkem má tedy analyzovaná funkce horní odhad časové složitosti $O(n)+5+O(n^2)=O(n^2)$.

Nyní přistupme k~analýze horního odhadu prostorové složitosti.
V~algoritmu jsou použity datové struktury $state, d, p$.
Každá z~nich obsahuje jednu položku pro každý uzel vstupního grafu, tedy mají prostorovou složitost $n+n+n=O(n)$.
Dále je použita struktura $Queue$, která bude obsahovat nanejvýš $n-1$ uzlů vstupního grafu a má tedy prostorovou složitost $O(n)$.
V~programu jsou také použit proměnné $u,v$, které mají prostorovou složitost $1+1=2$.
Celkem je horní odhad prostorové složitosti programu $O(n)+O(n)+2=O(n)$.


\section*{\centering{2. příklad}}
Problém $\rp$ je v~$NP$, jelikož lze sestrojit nedeterministický Turingův stroj, který sečte
váhy nedeterministicky zvolené podmnožiny a porovná je se součtem vah zbylých prvků množiny násobených zadaným poloměrem.
Důkaz $NP$-těžkosti a zároveň tedy i $NP$-úplnosti $\rp$ problému bude proveden redukcí z~problému $\sbs$ v~jeho definici z~přednášek.
Váhová funkce u~problému $\rp$ má jako obor hodnot definován obor $\mathbb{N}$ (předpokládejme, že $0 \in \mathbb{N}$), zatímco u~problému
$\sbs$ je obor hodnot definován pro $\mathbb{Z}$.
Pokud budeme předpokládat, že i problém $\sbs$ má vahovou funkci definovánu pro obor hodnot $\mathbb{N}$, tak důkaz NP-úplnosti
$\sbs$ v~podobě z~přednášek zůstává platný, neboť se v~tomto důkazu nepracuje se zápornými čísly jako hodnotami váhové funkce.
Proto dále může předpokládat, že problém $\sbs$ s~váhovou funkcí s~oborem hodnot $\mathbb{N}$ je také $NP$-úplný.

Redukce z~problému $\sbs$ na problém $\rp$ bude následující:
\begin{itemize}
\item Nechť $S$ je konečná množina položek problému $\sbs$.
Konečná množina položek $T$ problému\\ $\rp$ je pak rovna $S=T \cup \{z\}$,
kde $z \not \in S$.
\item Poměr $r=1$
\item Pro váhovou funkci $v$ problému $\rp$ platí:\\
        $$v=w\, \cup\, \{
    v(z) = 
        \begin{cases}
            2*W-w(S) &\mbox{if } W \geq (w(S)-W) \\
            w(S)-2*W & \mbox{if } W < (w(S)-W) 
        \end{cases}
\},$$
kde $w(S) = \sum_{s\in S} w(s)$, $w$ je váhová funkce a $W$ váha podmnožiny v~$\sbs$.
\end{itemize}

\section*{\centering{3. příklad}}
Dokažme, že platí $P=NP \Rightarrow \ltl$ je $NP$-úplný.
Pokud je $\ltl$ $NP$-úplný, tak $\ltl$ je $NP$-těžký a $\ltl \in P$.
Jazyk $\ltl$ lze přijímat deterministickým Turingovým strojem, který příjme právě řetězec $true$, a tudíž $\ltl \in P$.
Nyní dokažme, že $\ltl$ je $NP$-těžký.
Pokud je $\ltl$ $NP$-těžký tak platí, že $\forall L' \in NP: L' \leq \ltl$.
Je tedy třeba ukázat, že existuje polynomiální redukce z~libovolného jazyka $L' \in NP$ na jazyk $\ltl$.
Turingův stroj realizující polynomiální redukci by fungoval následujícím způsobem (předpokládejme TS $T_{L'}$ přijímající $L'$ a TS $T_{\ltl}$
přijímající $\ltl$):
\begin{itemize}
    \item Turingův stroj zapíše na pásku $\omega \in L'$ a simuluje běh stroje $T_{L'}$.
    \item Pokud simulovaný stroj přijal, zapíše na pásku řetězec $true$, jinak řetězec $false$.
    \item Simuluje stroj $T_{\ltl}$ a příjme, pokud tento příjme.
\end{itemize}
Jelikož $L' \in NP$ a dle předpokladu $P=NP$, tak první krok stroje realizujícího redukci bude v~$P$.
Druhý krok stroje má konstantní složitost a třetí je opět v~$P$, protože $\ltl \in P$.
Celkem má tedy Turingův stroj realizující redukci časovou složitost v~$P$.
Jelikož existuje polynomiální redukce z~libovolného $L' \in NP$ na $\ltl$, tak $\ltl$ je $NP$-těžký, a tedy také $NP$-úplný.

\section*{\centering{4. příklad}}
V~příkladu 3 bylo ukázáno, že lze provést redukci jazyka $L' \in NP$ v~polynomiálním čase na jazyk $\ltl \in P$.
Pokud bychom v~třetím kroku stroje z~minulého příkladu provádějícího redukci nahradili stroj $T_{\ltl}$ za stroj $T_L$ přijímající
nějaký jazyk $L\in NP$, tak by redukce zůstala polynomiální, jelikož předpokládáme $P=NP$, a tudíž $L$ je $NP$-těžký, a tedy také
$NP$-úplný.
\end{document}
